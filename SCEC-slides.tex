\documentclass{beamer}
\usepackage{hyperref}
\usepackage{subfig}  %% Para incluir subgraficos
\usepackage{graphicx}
\usepackage{media9} % 
\usepackage{url}
\usepackage{ragged2e}  % Allow justification
\usepackage[margin=20pt,font=small,labelfont=bf,labelsep=period]{caption}

\usepackage[spanish, activeacute]{babel}
\usepackage[utf8]{inputenc}
\decimalpoint
%\usepackage{natbib}

%
\hypersetup{pdfstartview={Fit}, bookmarks=True, pdftitle={SCEC Presentation}, pdfauthor={Juan Carlos Vergara-Gallego}, pdfsubject={Presentation}, pdfkeywords={Waves, Elasticity, Numerical Methods, High Performance Computing, BEM}, pdfpagemode=UseOutlines, bookmarks, bookmarksopen, pdfstartview=FitH, colorlinks,linkcolor=blue, urlcolor=black, citecolor=blue}  % Configure hyperref

%--- New commands ----%
\newcommand{\footref}[1]{\textsuperscript{\ref{#1}}}
\newcommand{\pardiff}[2]{\frac{\partial #1}{\partial #2}}
\newcommand{\pardiffd}[2]{\frac{\partial^2 #1}{\partial #2^2}}
%---------------------%

%\usefonttheme[onlymath]{serif}  % Make equations to be in serif fonts
\usefonttheme{serif}  % Make equations to be in serif fonts

\begin{document}


%title
\title[CyberShake] % (optional, only for long titles)
{A Physics-Based Seismic Hazard Model for Southern California}
\subtitle{CyberShake, Southern California Earthquake Center}
\author[Vergara Gallego, Juan Carlos] % (optional, for multiple authors)
{Juan Carlos Vergara Gallego\\ \texttt{\small jvergar2@eafit.edu.co}\\
{\tiny Presentación disponible en: \url{https://github.com/jvergar2/03_SCEC}}}
\institute{Departamento de Ingeniería Civil\\
  Universidad EAFIT}
\date{\today}
\subject{Ingeniería Sísmica}

% Title page
\frame{\titlepage}

% Outline
\begin{frame}
	\frametitle{Outline}
	\tableofcontents
\end{frame}
%
%
\section{Introduction}
\begin{frame}
\frametitle{Introducción}
%
\justifying
Se presenta un resumen del proyecto {C}yber{S}hake, de sus objetivos y la foma como están abordando el problema de construir el modelo de Amenaza en el Sur de California.

\url{http://scec.usc.edu/scecpedia/CyberShake}
%
\end{frame}
%
%
\begin{frame}[allowframebreaks]
\frametitle{¿Qué es el CyberShake?}
%
\justifying
{C}yber{S}hake, es un proyecto de investigación del ``Southern California Earthquake Center's" (SCEC), dentro del cual se encuentran desarrollando un modelo computacional a gran escala para incluir determinísticamente el efecto de la fuente y la ruta de propagación de las ondas sísmicas en la amenaza sísmica del Sur de California.
%
\end{frame}
%\transwipe
%
\begin{frame}
\frametitle{¿Qué es el CyberShake?}
\begin{figure}[h]
	\centering
	\includegraphics[height=6cm]{img/CyberShake_2009.PNG}
	\caption{Mapa de amenaza sísmica del Sur de California calculado con CyberShake. \footnote{\url{http://scec.usc.edu/scecwiki/images/6/61/CandB_2008.PNG}\\}}
\end{figure} 
%
\end{frame}
%
%
\begin{frame}
\frametitle{¿Qué es el CyberShake?}
%
\begin{figure}[h]
	\centering
	\includegraphics[height=6cm]{img/CandB_2008.PNG}
	\caption{Mapa de amenaza sísmica del Sur de California calculado con las ecuaciones de predición del movimiento del suelo (GMPE). \footnote{\url{http://scec.usc.edu/scecwiki/images/6/61/CandB_2008.PNG}\\}}
\end{figure}
%
%
%\begin{figure}[h]
%\centering
%%
%	\subfloat[Ricker pulse.]{\includegraphics[width=0.4\textwidth]{img/CyberShake_2009.PNG}}\qquad
%	%
%	\subfloat[Ricker pulse spectrum.]{\includegraphics[width=0.4\textwidth]{img/CandB_2008.PNG}}
%	%
%\caption{Ricker pulse and its spectrum.}
%
%\end{figure}
\end{frame}
%
%
\section{¿Qué se hace actualmente?}
\begin{frame}[allowframebreaks]\frametitle{Ground Motion Prediction Equations}
%
Acá voy a meter un texto que me va a pasar La Morsa.
%
%
\begin{figure}[h]
	\centering
	\includegraphics[height=3cm]{img/UCERF2_GMPE_2007.PNG}
	\caption{Mapa de amenaza sísmica del Sur de California calculado con cautro ecuaciones de predición del movimiento del suelo (GMPE) diferentes. \footnote{\url{http://scec.usc.edu/scecwiki/images/b/bd/UCERF2_GMPE_2007.PNG}\\}}
\end{figure}
%
A pesar de que las cuatro leyes de atenuación son aceptadas en la comunidad científica, es evidente las grandes diferencias entre ellas.
%
\end{frame}
%
%
\section{¿Con que información cuentan?}
\subsection{UCERF 2.0}
\begin{frame}[allowframebreaks]
\frametitle{Uniform California Earthquake Rupture Forecast, Version2.0 (UCERF 2.0)}
%
\justifying
El $UCERF2.0$ estima la probabilidad de ocurrencia de sismos con magnitud mayor o igual a $5$ $(M_W \geq 5.0)$ que pueden ocurrir dentro de una ventana de tiempo en California EE.UU.\\
%
Cuantro componentes básicos de $UCERF2.0$:
%
	\begin{itemize}
	\justifying
	%
		\item Modelo de Falla: Geometría física de las fallas conocidas.
		%
		\item Modelo de Deformación: ``Taza de deslizamiento" de cada sección de la falla Con eso se calcula el momento sísmico.
		%
		\item ``Earthquake Rate Model": Taza de todos los sismos dentro de una región sobre una magnitud mínima especificada.
		%
		\item Modelos de Probabilidad: Determina la probabilidad de ocurrencia de cada evento dentro de una ventana de tiempo.
	%
	\end{itemize}
%
\begin{figure}[h]
	\centering
	\includegraphics[height=1.75cm]{img/Components_UCERF.png}
	\caption{Componentes básicos del modelo UCERF2.0. \cite[figura 3, página 2057]{gravesetal}}
\end{figure}
%
%
Con la información suministrada por $UCERF2.0$ se identifican todas las posibles fallas sísmicas $200$ $km$ dentro de la región de estudio. Todas las fallas se usan para generar diferentes escenarios, dentro de los cuales se varía la ubicación del hypocentro y la forma de la ruptura.\\
%
En total se generan al rededor de $415.000$ escenarios de ruptura para cada sitio. 
%
%
\end{frame}
%
%
\subsection{Modelos de Velocidad}
\begin{frame}[allowframebreaks]
\frametitle{SCEC Community Velocity Model: CVM-S}
%
\justifying
Este modelo es una chimba
%
\end{frame}
%
%
\begin{frame}[allowframebreaks]{References}
\def\newblock{}
%\bibliographystyle{gji}
%\bibliography{refSCEC}
\bibliographystyle{plain}
\begin{thebibliography}{1}

\bibitem{book:aki} Keiiti Aki \& Paul G. Richards. {Q}uantitative {S}eismology. University Sciencie Books, 2nd Edition, Mill Valley, San Diego, 2002.

\bibitem{gravesetal} Graves, R.; Jordan, T. H.; Callaghan, S.; Deelman, E.; Field, E.; Juve, G.; ... \& Vahi, K. (2011). CyberShake: A physics-based seismic hazard model for southern California. Pure and Applied Geophysics, 168(3-4), 367-381.

\bibitem{wiki:FVM} Finite volume method. (2014, April 22). In Wikipedia, The Free Encyclopedia. Retrieved 00:44, September 25, 2014, from \url{http://en.wikipedia.org/w/index.php?title=Finite_volume_method&oldid=605282055}

\end{thebibliography}

\end{frame}


\end{document}

